\documentclass[12pt,a4]{article}
\usepackage[top=.75in, bottom=.75in, left=.75in, right=.75in]{geometry}
\usepackage{titlesec}

\titleformat*{\subsection}{\large\it\bfseries}
\titlespacing*{\section}{0pt}{6ex}{0ex}

\begin{document}

\thispagestyle{empty}

\noindent{\LARGE\textbf{Practical and Ethical Scenarios}}

\vspace{1em} 
\noindent Imagine yourself as the researcher in each of the following scenarios. What is ethical, responsible, and methodologically necessary in each situation? {\em Note: There may not be a right answer.}

\begin{enumerate}\itemsep1em

\item The researcher is interested in how the experience of pain affects individuals' physiological responses to threatening stimuli. To study this, the researcher randomly assigns individuals to a painful experience (holding their bare hand in a bowl of ice water for 10 minutes) or a control condition before exposing them to videos containing images of warfare and violence. % Benefit/harm trade-offs

\item The researcher wants to know if mosquito nets help reduce the spread of malaria in a Sub-Saharan Africa country, which prior research suggests are effective but the size of the effect is ambiguous. Some regions in the country have many malaria-carrying mosquitos, while some regions have few. The researcher randomly assigns households across all regions to receive mosquito nets or not. % ethics of randomization

\item The researcher is interested in whether giving grants to local libraries for community outreach initiatives increases use of the libraries. Libraries are randomly assigned to receive either no grant, a small grant (20,000 kr), or a large grant (100,000 kr). After the study is over, the researcher learns that due to an administrative error, several libraries that were supposed to receive large grants received no grants, and vice versa. % ethical violations

\item All details are the same as the previous scenario, but the ``administrative error'' was made intentionally by the spouse of a local library administrator.

\item The researcher is interested in whether international study experiences by high school students affects the completion time for their post-secondary degrees (either positively or negatively). One thousand students at a university are invited to participate in the study, which will use a lottery to allocate scholarships for stays abroad. Twenty of the students who do not receive a scholarship in the lottery apply elsewhere for a scholarship and eventually study go abroad. % compensatory behavior

\item The researcher is interested in testing a game theoretic expectation about levels of social trust after repeated interpersonal interactions. Individuals randomly two conditions play a ``trust game'' with another participant for ten rounds, but individuals in the treatment condition are surprised by a eleventh round where they can potentially lose funds to the other participant, while those in the control condition proceed directly to an outcome questionnaire. % Deception: Omission

% \item % Deception: Commission

\item The researcher's goal is to study the effects of bribe-taking on local governments' responsiveness to citizen concerns. The researcher randomly assigns municipalities to be in ``bribe'' or ``no bribe'' conditions, surveys residents in each municipality about their concerns, encourages anyone with a concern to contact a local official. For those living in ``bribe'' municipalities, the researcher also supplies a modest sum of money and instructs the residents to offer the bribe.

\item The researcher is interested in the effect of an early childhood intervention on individuals' wages in adulthood and plans to follow a group of children for 30 years. Knowing that the study will be difficult, the researcher plans to follow 500 individuals. After ten years, 80\% of treatment group participants were reinterviewed but only 60\% of control group participants were able to be interviewed. % Attrition

\item The researcher designs a 2-by-2 factorial design crossing the source of a persuasive argument and the supportive or opposed argument made in the text in order to study the effect of communication on opinions. After the study, the researcher finds the source of argument has no detectable effect on opinions, so instead focuses on an interaction between subjects' ideology and the pro/con direction of the text. % hypotheses not from the data

\item The researcher is interested in how responsive government officials are to individuals to immigrant and native-born citizens and suspects that members of parliament are more likely to respond and help Danes than immigrants. The researcher randomly assigns them to receive letters requesting assistance from individuals with either Arabic-sounding names or Danish-sounding names in order to measure and record their responses. The researcher publishes the results in a major newspaper editorial that criticizes the government. % Informed consent

\item The researcher works for a political party and is trying to design effective campaign materials for their party organization to use in the next election. The researcher designs several versions of campaigns slogans and constructs a representative sample of the population, with each individual in the sample randomly assigned to a different slogan. The researcher then places phone calls to each individual describing the survey interview as an independent public opinion poll. The researcher uses the results of the study to design the next election campaign. % informed consent

\item The researcher's goal is to find the most cost-effective means of providing long-term health and social care to individuals with severe mental and physical disabilities. The researcher randomly assigns individuals in this subpopulation are randomly assigned to live in either a small residential setting or a large, hospital-like setting. The researcher finds the large, hospital-like setting to be the most cost-effective. % Vulnerable population

\item The researcher is interested in whether ticket enforcement on municipal buses is cost effective (i.e., whether the cost of paying ticket enforcers is outweighed by the gain from more riders paying for their tickets). The researcher believes that ticket violations (i.e., not buying a ticket) are highest in the city's ten districts with the lowest median income. The researcher randomly assigns five of these districts to receive increased enforcement. %Differential enforcement across regions; generalizability to other regions

\item The researcher wants to know if leadership training for ministerial managers improves the productivity of their employees. The training program's effectiveness will be assessed every six months and continue for three years. After 12 months the program shows a modest, positive effect but it is hard to distinguish from no gain. After 18 months, the effect appears to be smaller but still positive. The ministry receives additional funding for the program and suggests to the researcher that the study be discontinued and the program expanded to all managers. % Discontinuing a study early: apparent effectiveness

\item The researcher believes that previously documented evidence of the effect of political advertising on support for candidates is overstated (i.e., that the effect is smaller than extant evidence). The researcher determines they need 1000 experimental participants to have sufficient power to detect an effect size half the size of extant evidence. After 800 participants have completed the experiment, the effect is not distinguishable from zero. The researcher decides to stop collecting data in order to save funds for another project. % Discontinuing a study early: apparent ineffectiveness

\item The researcher is interested in the effect of intergroup contact on evaluations of in-group and out-group individuals. The researcher plans to assign participants to work in teams on a short task, with half of the teams homogeneous (made up of all participants from the same ethnic group) or heterogeneous (made up of a mix of ethnic groups) and measure individuals' thermometer ratings of members of each ethnic group. Before the study, the researcher learns that many of the study participants have already participated in a year-long intergroup contact intervention run by a community organization. % Pretreatment

\item The researcher is interested in whether a new primary school math curriculum increases student math performance. Schools are selected for participation in the study and teachers are randomly assigned to either the new curriculum or the old curriculum to use in the next school year. While happy to participate in the study, after receiving their assignments several teachers express concerns that being assigned to their less preferred curriculum means they won't teach as well as they otherwise would. % treatment preferences

\item The researcher is interested in whether the time of day that hospital doctors work (morning, afternoon, evening) affects affects medical error rates. Doctors are told they will participate in an experiment where their working hours are randomized. Doctors are asked for their preferred working shift. Forty percent say they do not have a preference, while sixty percent have a strong preferences for one of the three options. The researcher uses only those expressing no preference in the study. % treatment preferences

\item The researcher is studying teamwork and is wants to know whether teams work better when run by a pre-assigned leader or by a leader elected by the team members. In order to save costs, participants in this study come from a university-run subject pool, with participants completing many studies over-time. When the study is scheduled, the researcher learns that participants will have completed a study about the effects of lying on individuals' social trust immediately before entering the laboratory for this study. % Subject pool contamination

\item The researcher is interested in the effect of microfinance loans on child mortality rates in a low-income country. The researcher randomly allocates villages to participate in the loan program and then allocates loans within treatment-group villages through a random lottery. Several individuals in the treatment-group villages who did not receive loans live in the same households as individuals who received loans. The researcher is unable to detect a difference between treatment and control group households in the treatment villages, but finds a significant effect when using the control villages as a baseline for comparison. % Interference/SUTVA violations

\end{enumerate}

\end{document}